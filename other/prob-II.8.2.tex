\documentclass[11pt,letterpaper]{article}
\usepackage[left=1.00in, right=1.00in, top=1.00in, bottom=1.00in]{geometry}

\usepackage{mathrsfs}
\usepackage{mathtools}

\usepackage{amsmath}
\usepackage{amsfonts}
\usepackage{amssymb}

\begin{document}

\noindent
Math W80 Homework \hfill Micah Ng \hfill \today

\begin{center}
\line(1, 0){250}
\end{center}

\begin{enumerate}
\item[II.8.2]
  \begin{enumerate}
  \item
    \begin{align*}
      q(x,y,z) &= 2x^2+5y^2+2z^2+2xz \\
          &= \begin{bmatrix}x & y & z\end{bmatrix}
             \begin{bmatrix}
               2 & 0 & 1 \\
               0 & 5 & 0 \\
               1 & 0 & 2
             \end{bmatrix}
             \begin{bmatrix}x \\ y \\ z\end{bmatrix}
    \end{align*}

  \item
    \begin{align*}
      |A_1| &= \begin{vmatrix}2\end{vmatrix} = 2 > 0 \\
      |A_2| &= \begin{vmatrix}2 & 0 \\ 0 & 5 \end{vmatrix} = 10 > 0 \\
      |A_3| &= 
             \begin{vmatrix}
               2 & 0 & 1 \\
               0 & 5 & 0 \\
               1 & 0 & 2
             \end{vmatrix}
            = 15 > 0
    \end{align*}
    Thus by theorem 8.8, $q$ is positive definite.

  \item
    \begin{align*}
      0 &= \begin{vmatrix}
             2-\lambda & 0 & 1 \\
             0 & 5-\lambda & 0 \\
             1 & 0 & 2-\lambda
           \end{vmatrix} \\
        &= (5-\lambda)((2-\lambda)(2-\lambda)-1) \\
        &= (5-\lambda)\left(3-4\lambda+\lambda^2\right) \\
        &= (5-\lambda)(3-\lambda)(1-\lambda)
    \end{align*}
    Eigenvalues, $1,3,5$, are all positive, so $q$ is positive definite.
  \end{enumerate}
\end{enumerate}

\begin{center}
\line(1, 0){250}
\end{center}


\end{document}
