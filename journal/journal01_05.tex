\documentclass[11pt,letterpaper]{article}
\usepackage[left=1.00in, right=1.00in, top=1.00in, bottom=1.00in]{geometry}

\usepackage{mathrsfs}
\usepackage{mathtools}

\usepackage{amsmath}
\usepackage{amsfonts}
\usepackage{amssymb}

\begin{document}

\noindent
Math W80 Journal \hfill Micah Ng \hfill \today

\begin{center}
\line(1, 0){250}
\end{center}

Sections 7 of chapter 1 felt mostly like a generalization of limits as
described in past classes. The main difference being the extensions to support
computing limits of functions whose input and output can be vectors. These
similarities made reading relatively easy. Vector inputs did increase the
difficulty of epsilon-delta limit proofs from what I have worked with single
variables.

The proofs at the end of the section confused me with the appearance of $s$ and
$p$ functions, until I found the actually were defined after the proofs.

I am unsure what the book means when stating ``$\mathbf{a}$ is not a limit
point''.

\begin{center}
\line(1, 0){250}
\end{center}


\end{document}
