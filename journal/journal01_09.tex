\documentclass[11pt,letterpaper]{article}
\usepackage[left=1.00in, right=1.00in, top=1.00in, bottom=1.00in]{geometry}

\usepackage{mathrsfs}
\usepackage{mathtools}

\usepackage{amsmath}
\usepackage{amsfonts}
\usepackage{amssymb}

\begin{document}

\noindent
Math W80 Journal \hfill Micah Ng \hfill \today

\begin{center}
\line(1, 0){250}
\end{center}

Section 1 of chapter 2 begins the generalization of derivatives to multiple
variables. The line integral intuitively follows the equivalent rules of sum,
product, and composition, modified for vector output.

On the last page, the sequence of steps leading up to $df=\frac{df}{dx}dx$
individually made sense (though the result felt like cheating). However, I am
unable to conceptualize what this result actually implies.

\begin{center}
\line(1, 0){250}
\end{center}


\end{document}
