\documentclass[11pt,letterpaper]{article}
\usepackage[left=1.00in, right=1.00in, top=1.00in, bottom=1.00in]{geometry}

\usepackage{mathrsfs}
\usepackage{mathtools}

\usepackage{amsmath}
\usepackage{amsfonts}
\usepackage{amssymb}

\begin{document}

\noindent
Math W80 Journal \hfill Micah Ng \hfill \today

\begin{center}
\line(1, 0){250}
\end{center}

The beginning of section 8 of chapter 1 was not too difficult, as it mostly
reviewed the topics earlier discussed in class.

I found the proof of theorem 8.8 clever in the use of the sequence
$b-\frac{1}{n}$ to converge (from within the set $f(D)$) to the least upper
bound, showing that $b\in f(D)$ since a closed set contains all its limit
points.

The book states that $f:D\to\mathscr{R}$ is uniformly continuous if given
$\varepsilon>0$, there exists $\delta>0$ such that
$\mathbf{x},\mathbf{y}\in D$, $|\mathbf{x}-\mathbf{y}|<\delta$ implies
$|f(\mathbf{x})-f(\mathbf{y})|<\varepsilon$. This is nearly the same definition
as for continuous, with the only difference being that $\mathbf{y}$ is not
fixed, however I am having trouble visualizing how this would change the
result.

I am unsure what theorem 8.10 is asserting due to the book's wording and
notation.

\begin{center}
\line(1, 0){250}
\end{center}


\end{document}
