\documentclass[11pt,letterpaper]{article}
\usepackage[left=1.00in, right=1.00in, top=1.00in, bottom=1.00in]{geometry}

\usepackage{mathrsfs}
\usepackage{mathtools}

\usepackage{amsmath}
\usepackage{amsfonts}
\usepackage{amssymb}

\begin{document}

\noindent
Math W80 Journal \hfill Micah Ng \hfill \today

\begin{center}
\line(1, 0){250}
\end{center}

Unlike section 6 of chapter 2, section 7 is fully new (and quite complicated).
Although I have previously seen the binomial coefficient,
${k\choose j}=\frac{k!}{j!(k-j)!}$, the multinomial $k\choose j_1\cdots j_n$ is
new, and feels strange since the denominator consists only of a product of the
factorials of the $j_i$'s.

In the formula for $P_k(\mathbf{h})$, I am unsure where the $j_i+\cdots+j_n=k$
(from the sum) comes from. I can compute it, but I do not know why.

The critical point classifier we discussed in class, using the Hessian matrix,
seems convenient (though intensive if many variables are in play), but I wonder
how common it is to run into quadratic forms.

\begin{center}
\line(1, 0){250}
\end{center}


\end{document}
