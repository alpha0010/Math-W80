\documentclass[11pt,letterpaper]{article}
\usepackage[left=1.00in, right=1.00in, top=1.00in, bottom=1.00in]{geometry}

\usepackage{mathrsfs}
\usepackage{mathtools}

\usepackage{amsmath}
\usepackage{amsfonts}
\usepackage{amssymb}

\begin{document}

\noindent
Math W80 Journal \hfill Micah Ng \hfill \today

\begin{center}
\line(1, 0){250}
\end{center}

The reading of sections 1-6 of chapter 1 was long. I knew it was long, but the
denseness of the writing style greatly slowed my reading speed, making it take
significantly longer than expected.

Sections 1 and 2 were essentially all review, and easy to understand. Although,
they did make me realize the extent to which I have forgotten my past linear
algebra. The concepts of section 3 were also mostly review. Reorienting my view
took some doing since this book uses $\langle\mathbf{x},\mathbf{y}\rangle$ for
inner/dot products. Examples, including polynomial vector space $\mathscr{P}$
and Fourier series, were new to me.

Sections 4-6 were harder. Complicated notation made some of the logic difficult
to follow (not helped by the frequent use of words like ``clearly'' and
``simply''). I have reasonable understanding of the basics of mappings,
kernels, images, and determinants, but cannot commit to memory, nor likely
solve problems involving, the plethora of theorems in the later end of each
of these section.

\begin{center}
\line(1, 0){250}
\end{center}


\end{document}
