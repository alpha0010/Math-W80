\documentclass[11pt,letterpaper]{article}
\usepackage[left=1.00in, right=1.00in, top=1.00in, bottom=1.00in]{geometry}

\usepackage{mathrsfs}
\usepackage{mathtools}

\usepackage{amsmath}
\usepackage{amsfonts}
\usepackage{amssymb}

\begin{document}

\noindent
Math W80 Journal \hfill Micah Ng \hfill \today

\begin{center}
\line(1, 0){250}
\end{center}

Chapter 2 section 8 discusses classification of critical points in
$\mathbb{R}^n$. This process is significantly longer than the previously
covered lower dimensional procedures, especially when constraints are added
to the function.

It has been a long time since I last saw eigenvalues, so that took some
reviewing to get me up to speed.

\begin{center}
\line(1, 0){250}
\end{center}


\end{document}
