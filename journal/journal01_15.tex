\documentclass[11pt,letterpaper]{article}
\usepackage[left=1.00in, right=1.00in, top=1.00in, bottom=1.00in]{geometry}

\usepackage{mathrsfs}
\usepackage{mathtools}

\usepackage{amsmath}
\usepackage{amsfonts}
\usepackage{amssymb}

\begin{document}

\noindent
Math W80 Journal \hfill Micah Ng \hfill \today

\begin{center}
\line(1, 0){250}
\end{center}

Section 5 of chapter 2 is complicated. Most of the examples required several
readings before I understood their processes. I am unsure if I have grasped the
details of each proof.

From what I can tell, a manifold is an object of $<n$ dimensions in
$\mathbb{R}^n$ such that every point on it has a tangent (hyper) plane. This
object is further dividable into patches. I was not able to tell from the
definition if manifolds must be bounded (as is the case for the provided
examples).

The author worries me by utilizing the implicit function theorem and the
implicit mapping theorem, but stating their proofs lie in the next chapter.
Since each section builds off the previous, pulling from the future like this
seems to be a risk for circular logic.

\begin{center}
\line(1, 0){250}
\end{center}


\end{document}
