\documentclass[11pt,letterpaper]{article}
\usepackage[left=1.00in, right=1.00in, top=1.00in, bottom=1.00in]{geometry}

\usepackage{mathrsfs}
\usepackage{mathtools}

\usepackage{amsmath}
\usepackage{amsfonts}
\usepackage{amssymb}

\begin{document}

\noindent
Math W80 Homework \hfill Micah Ng \hfill \today

\begin{center}
\line(1, 0){250}
\end{center}

\noindent
Homework 11

\begin{enumerate}
\item[28.]
  \begin{align*}
    f(x,y,z) &= x^2+2y-z^2 \\
    g_1(x,y,z) &= 2x-y \\
    g_2(x,y,z) &= y+z \\
    \nabla f &= \begin{bmatrix}2x \\ 2 \\ -2z\end{bmatrix} \\
    \nabla g_1 &= \begin{bmatrix}2 \\ -1 \\ 0\end{bmatrix} \\
    \nabla g_2 &= \begin{bmatrix}0 \\ 1 \\ 1\end{bmatrix} \\
    2x &= 2\lambda_1 \\
    2 &= -\lambda_1+\lambda_2 \\
    -2z &= \lambda_2 \\
    0 &= 2x-y \\
    0 &= y+z
  \end{align*}
  According to WolframAlpha, this results in a critical point at
  $\left(\frac{2}{3},\frac{4}{3},-\frac{4}{3}\right)$ with
  $\lambda_1=\frac{2}{3},\lambda_2=\frac{8}{3}$.
  \begin{align*}
    &\begin{bmatrix}
      2 & -1 & 0 \\
      0 & 1 & 1
    \end{bmatrix} \\
    &\begin{bmatrix}
      2 & 0 & 1 \\
      0 & 1 & 1
    \end{bmatrix} \\
    &\begin{bmatrix}
      1 & 0 & \frac{1}{2} \\[0.3em]
      0 & 1 & 1
    \end{bmatrix} \\
    \begin{bmatrix}x \\ y \\ z\end{bmatrix}
      &= s\begin{bmatrix}
            -\frac{1}{2} \\[0.3em]
            -1 \\
            1
          \end{bmatrix}
  \end{align*}
  \begin{align*}
    \nabla h &= \nabla f-\lambda_1\nabla g_1-\lambda_2\nabla g_2 \\
        &= \nabla f-\frac{2}{3}\nabla g_1-\frac{8}{3}\nabla g_2 \\
        &= \begin{bmatrix}
             2x-\frac{4}{3} \\[0.3em]
             2+\frac{2}{3}-\frac{8}{3} \\[0.3em]
             -2z-\frac{8}{3}
           \end{bmatrix} \\
    H_h &= \begin{bmatrix}
             2 & 0 & 0 \\
             0 & 0 & 0 \\
             0 & 0 & -2
           \end{bmatrix} \\
    q\left(-\frac{s}{2},-s,s\right) &=
        \begin{bmatrix}-\frac{s}{2} & -s & s\end{bmatrix}
        \begin{bmatrix}
          2 & 0 & 0 \\
          0 & 0 & 0 \\
          0 & 0 & -2
        \end{bmatrix}
        \begin{bmatrix}-\frac{s}{2} \\[0.3em] -s \\ s\end{bmatrix} \\
        &= 2\left(-\frac{s}{2}\right)^2-2s^2 \\
        &= -\frac{3}{2}s^2
  \end{align*}
  So $q$ is negative definite on the tangent space, thus
  $\left(\frac{2}{3},\frac{4}{3},-\frac{4}{3}\right)$ is a local maximum.
\end{enumerate}

%\newpage
\begin{center}
\line(1, 0){250}
\end{center}

\noindent
ACoSV II

\begin{enumerate}
\item[8.2]
  \begin{enumerate}
  \item
    \begin{align*}
      q(x,y,z) &= 2x^2+5y^2+2z^2+2xz \\
          &= \begin{bmatrix}x & y & z\end{bmatrix}
             \begin{bmatrix}
               2 & 0 & 1 \\
               0 & 5 & 0 \\
               1 & 0 & 2
             \end{bmatrix}
             \begin{bmatrix}x \\ y \\ z\end{bmatrix}
    \end{align*}

  \item
    \begin{align*}
      |A_1| &= \begin{vmatrix}2\end{vmatrix} = 2 > 0 \\
      |A_2| &= \begin{vmatrix}2 & 0 \\ 0 & 5 \end{vmatrix} = 10 > 0 \\
      |A_3| &= 
             \begin{vmatrix}
               2 & 0 & 1 \\
               0 & 5 & 0 \\
               1 & 0 & 2
             \end{vmatrix}
            = 15 > 0
    \end{align*}
    Thus by theorem 8.8, $q$ is positive definite.

  \item
    \begin{align*}
      0 &= \begin{vmatrix}
             2-\lambda & 0 & 1 \\
             0 & 5-\lambda & 0 \\
             1 & 0 & 2-\lambda
           \end{vmatrix} \\
        &= (5-\lambda)((2-\lambda)(2-\lambda)-1) \\
        &= (5-\lambda)\left(3-4\lambda+\lambda^2\right) \\
        &= (5-\lambda)(3-\lambda)(1-\lambda)
    \end{align*}
    Eigenvalues, $1,3,5$, are all positive, so $q$ is positive definite.
  \end{enumerate}

\item[8.5]
\end{enumerate}

\begin{center}
\line(1, 0){250}
\end{center}


\end{document}
