\documentclass[11pt,letterpaper]{article}
\usepackage[left=1.00in, right=1.00in, top=1.00in, bottom=1.00in]{geometry}

\usepackage{mathrsfs}
\usepackage{mathtools}

\usepackage{amsmath}
\usepackage{amsfonts}
\usepackage{amssymb}

\begin{document}

\noindent
Math W80 Homework \hfill Micah Ng \hfill \today

\begin{center}
\line(1, 0){250}
\end{center}

\noindent
Homework 3

\begin{enumerate}
\setcounter{enumi}{7}
\item
  \begin{enumerate}
  \item
    The limit points are the closed ball of radius $r$ about the origin.

  \item
    The only limit point is 0.

  \item
    There are no limit points.

  \item
    The limit points are $\mathbb{R}^n$.
  \end{enumerate}

\item
  \begin{enumerate}
  \item True.
  \item True.
  \item True.
  \item True.
  \item True.
  \item True.
  \item False.
  \end{enumerate}

\item
  \begin{enumerate}
  \item
    $\mathbf{a}$ must be a limit point: $B_r(\mathbf{a})$ must contain at least
    one point, distinct from $\mathbf{a}$, in $D$ for every $r>0$.

  \item
    $\forall\varepsilon>0,\exists\delta>0$ such that
    $0<|\mathbf{x}-\mathbf{a}|<\delta$ implies $|f(\mathbf{x})-L|<\varepsilon$.

  \item
    $\forall\varepsilon>0,\exists\delta>0$ such that
    $\mathbf{x}\in B_\delta(\mathbf{a}),\mathbf{x}\neq\mathbf{a}$ implies
    $f(\mathbf{x})\in B_\varepsilon(\mathbf{L})$.
  \end{enumerate}

\item
  \begin{enumerate}
  \item Closed.
  \item Both open and closed.
  \item Neither open nor closed.
  \item Open.
  \item Closed.
  \item Neither open nor closed.
  \item Open.
  \item Closed.
  \item Closed.
  \end{enumerate}

\item Compact sets are: b, e, h, i

\item
  \begin{enumerate}
  \item
    $|\mathbf{x}_2-\mathbf{x}_1|$, $\sqrt{(x_2-x_1)^2+(y_2-y_1)^2}$

  \item
    $|\mathbf{x}_2-\mathbf{x}_1|$, $\sqrt{(x_2-x_1)^2+(y_2-y_1)^2+(z_2-z_1)^2}$
  \end{enumerate}
  In $\mathbb{R}^n$: $|\mathbf{x}_2-\mathbf{x}_1|$,
  $\sqrt{\sum_{i=1}^{n}(x_{2,i}-x_{1,i})^2}$

\item
  $f=s\circ(p\circ(\pi_1,\pi_1),p\circ(2,\pi_1,\pi_3,\pi_3)$ \\
  Since $f$ is composed of continuous functions, $f$ is continuous.

\end{enumerate}

\newpage
\noindent
ACoSV

\begin{enumerate}
\item[I.7.6]
  Let $\mathbf{a}$ be a boundary point of $D\subset\mathbb{R}^n$. Then either
  $\mathbf{a}\in D$ or $\mathbf{a}\notin D$. \\
  Suppose $\mathbf{a}\in D$. Then we are done. \\
  Suppose $\mathbf{a}\notin D$. Then since $\mathbf{a}$ is a boundary point,
  for every $r>0$, $B_r(\mathbf{a})$ contains at least one point of $D$. So
  $\mathbf{a}$ is a limit point. \\
  Thus every boundary point of $D$ is either in $D$, or a limit point of $D$.

\item[I.8.4]
  Suppose $f:\mathbb{R}^n\to\mathbb{R}^m$, and the inverse image
  $f^{-1}(U)=\{\mathbf{x}\in\mathbb{R}^n|f(\mathbf{x})\in U\}$ is open for any
  open set $U\subset\mathbb{R}^m$.
\end{enumerate}

\begin{center}
\line(1, 0){250}
\end{center}


\end{document}
