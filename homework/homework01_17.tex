\documentclass[11pt,letterpaper]{article}
\usepackage[left=1.00in, right=1.00in, top=1.00in, bottom=1.00in]{geometry}

\usepackage{mathrsfs}
\usepackage{mathtools}

\usepackage{amsmath}
\usepackage{amsfonts}
\usepackage{amssymb}

\begin{document}

\noindent
Math W80 Homework \hfill Micah Ng \hfill \today

\begin{center}
\line(1, 0){250}
\end{center}

\noindent
Homework 10

\begin{enumerate}
\setcounter{enumi}{25}
\item
  \begin{enumerate}
  \item
    \begin{align*}
      Q(x,y) &= \mathbf{x}^\top\mathbf{A}\mathbf{x} \\
          &= \begin{bmatrix} x & y \end{bmatrix}
             \begin{bmatrix} a & b \\ c & d \end{bmatrix}
             \begin{bmatrix} x \\ y \end{bmatrix} \\
          &= \begin{bmatrix} ax+cy & bx+dy \end{bmatrix}
             \begin{bmatrix} x \\ y \end{bmatrix} \\
          &= ax^2+cxy+bxy+dy^2 \\
          &= ax^2+(b+c)xy+dy^2
    \end{align*}
    Thus $Q(x,y)$ 0\textsuperscript{th}- nor 1\textsuperscript{st}-degree
    terms.

  \item
    $Q(x,y)=y^2$ is neither positive definite, nor negative definite, nor
    nondefinite, so taking its coefficients and the result from (a), one
    corresponding matrix is
    $\mathbf{A}=\left[\begin{smallmatrix} 0 & 2 \\ -2 & 1 \end{smallmatrix}\right]$.
  \end{enumerate}

\item
  \begin{enumerate}
  \item
    \begin{align*}
      f(x,y) &= ax^2+2bxy+cy^2 \\
      f_x &= 2ax+2by \\
      f_{xx} &= 2a \\
      f_{xy} &= 2b \\
      f_y &= 2bx+2cy \\
      f_{yy} &= 2c \\
      \mathbf{H}_f(x,y) &= \begin{bmatrix}
                             2a & 2b \\
                             2b & 2c
                           \end{bmatrix}
    \end{align*}

  \item
    \begin{align*}
      q(x,y) &= \begin{bmatrix} x & y \end{bmatrix}
                \begin{bmatrix} 2a & 2b \\ 2b & 2c \end{bmatrix}
                \begin{bmatrix} x \\ y \end{bmatrix} \\
          &= \begin{bmatrix} 2ax+2by & 2bx+2cy \end{bmatrix}
             \begin{bmatrix} x \\ y \end{bmatrix} \\
          &= 2ax^2+2bxy+2bxy+2cy^2 \\
          &= 2ax^2+4bxy+2cy^2
    \end{align*}
  \end{enumerate}
\end{enumerate}

%\newpage
\begin{center}
\line(1, 0){250}
\end{center}
\noindent
ACoSV II

\begin{enumerate}
\item[7.3]

\end{enumerate}

\begin{center}
\line(1, 0){250}
\end{center}


\end{document}
